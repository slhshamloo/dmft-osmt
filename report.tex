\documentclass[titlepage, 11pt, a4paper]{article}
\usepackage[T1]{fontenc}
\usepackage{graphicx}
\usepackage{float}
\usepackage{tcolorbox}
\usepackage{amsmath}
\usepackage{amsfonts}
\usepackage{physics}
\usepackage{caption}
\usepackage{subcaption}

%Instructions : 5 pages, maximum 8 including figures

\newcommand{\cdag}[1]{\hat{c}_{#1}^{\dagger}}
\renewcommand{\c}[1]{\hat{c}_{#1}}
\newcommand{\n}[1]{\hat{n}_{#1}}
\newcommand{\ddfrac}[2]{\frac{\displaystyle #1}{\displaystyle #2}}

\begin{document}

\section{Introduction}

Dynamical Mean-Field theory (DMFT) has proven to be a fantastic tool to study the Hubbard model and Mott-Hubbard transitions. 
Still, the nature of Mott-Hubbard transitions is complex as we get away from the most simple models. Notably, in the years 2003 to 2005, a controversy was sparked regarding the existence of Orbital-Selective Mott Transitions (OSMTs). 
Indeed, in many transition metal oxides, multiple bands typically cross the Fermi surface, notably the $t_{2g}$ or $e_g$ orbitals, and different bandwidths for different bands might give rise to multiple transition for each bands. 
This is the case for the cuprate ${\mathrm{Ca}}_{2-x}{\mathrm{Sr}}_x{\mathrm{RuO}}_4$ which undergoes a Mott transition as $x$ is increased from 0 \cite{Nakatsuji2000}.

The model considered in this paper will be the case of two bands, one narrow ($W = 2$)  and one wide band ($W = 4$). The Hamiltonian used will be the Hubbard hamiltonian with interband coupling accounting for Hund's exchange :

\begin{align*}
H = &- \sum_{\langle i,j \rangle m \sigma} t_m \cdag{im\sigma} \c{jm\sigma} \\
&+ U \sum_{i m \sigma} \n{im\uparrow} \n{im\downarrow} + \sum_{i \sigma \sigma'} \left( U' - \delta_{\sigma \sigma'} J_z \right) \n{i1\sigma} \n{i2\sigma'} \\
&+ \frac{J_{\perp}}{2} \sum_{i m \sigma} \cdag{im\sigma} \left( \cdag{i \bar{m} \bar{\sigma}} \c{im\bar{\sigma}} + \cdag{im\bar{\sigma}} \c{i \bar{m} \bar{\sigma}} \right) \c{i \bar{m} \sigma}
\end{align*}

This model takes into account single-band Coulomb repulsion with the term in $U$, interband repulsion with $U' = U - 2J_z$ and Hund's exchange may be kept asymmetric ($J_z$ model) or simplified with $J_z = J_{perp} = J$ if the system is invariant under spin rotation ($J$ model). Sum indices denote sites ($i,j$), band ($m=1,2$) and spin ($\sigma = \uparrow, \downarrow$) with bars indicating the opposite state for the two states variables.

Initial results from Liebsch refuted the existence of OSMT for the $J_z$ model using a DMFT method with Quantum Monte-Carlo (QMC) solver at finite temperature \cite{Liebsch2003} \cite{Liebsch2004}. 
However, multiple papers later contradicted Liebsch's results and found an OSMT, first Koga et al. using DMFT with exact diagonalisation (ED) at zero temperature and the full $J$ model \cite{Koga2004}, arguing that $J\perp$ was necessary for OSMT observation \cite{Koga2005}, then Knecht et al. using DMFT with an improved QMC solver attempting to get rid of the sign problem at low temperature \cite{Knecht2005}, using the same $J_z$ model as Liebsch, and refuted both previous articles.
Finally Arita and Held also found an OSMT using projective QMC (PQMC) to tackle the $J$ model at zero temperature \cite{Arita2005}.
Liebsch has also published an answer to Knecht et al., comparing directly results from both articles \cite{Liebsch2005}.

\section{Methods}

Our goal in this paper is to use DMFT with a QMC solver to clarify the controversy and ensure the two-band Mott transition predictions is well understood and well described with current methods. 
To that end, we choose to study the full $J$ model, which supposedly encompasses the full effects of correlation, and is also more interesting as it has only been studied at 0 temperature, contrary to the $Jz$ model. 
In the Hamiltonian, $t_m$ is set to $1$ to set the energy scale. The parameters are chosen in accordance with previous papers on the matter, with the value of $U$ variable and the ratios $J = U/4$ and $U' = U - 2J = U/2$ fixed ; additionally the two bands are set with bandwidths $W_1 = 2 eV$ and $W_2 = 4 eV$ in an elliptical density of states given by $\rho_i (\epsilon) = \frac{4}{\pi W_i} \sqrt{1 - 4 \epsilon^2 / W_i^2}$, and taken at half filling.
Finally temperature is set to $\beta = \frac{1}{40} eV$, about room temperature.

In a Mott transition, the system undergoes a phase transition between a disordered phase for \(U<<t\), where the system is a conductor, and an ordered phase for \(U>>t\), where the Coulomb repulsion prevents electron hopping and the system becomes an insulator. The difference between the two phases is striking when looking at the spectral function. In the insulating phase, all the spectral weight is contained in two lobes separated by a (Mott) gap (figure \ref{fig:mott_transition_example}). The non zero quasi-particle weight at zero frequency is a signature of the conductive phase: there are available states right above the last filled one and therefore no gap between the occupied and unoccupied states. Hence, to characterize the Mott transition, the parameter of choice is the quasi-particle weight at the Fermi energy, obtained from the self-energy:
	\begin{equation}
		Z_i = \ddfrac{1}{1 - \eval{\pdv{\omega} \real(\Sigma_i)}_{\omega = 0}}
	\end{equation}
	that describes the width of the quasi-particle peak at \(\omega=0\). As the quasi-particle weight goes to zero, so does the spectral weight at $\omega = 0$, indicating an insulating state.
	\begin{figure}[h!]
		\centering
		\includegraphics[width=0.45\textwidth]{figures/spectral_before_vs_after.pdf}
		\caption{Spectral function of the narrow band \(W=2\)eV in the conducting (blue) and the Mott insulator (red) phase, computed using DMFT. The spectral weight at \(\omega=0\) allows to characterize the phase of the system.}
		\label{fig:mott_transition_example}
	\end{figure}
For approximating the quasiparticle weight, we used a discrete approximation from the first Matsubara frequency. As the first Matsubara frequency is given by $\omega_0 = \pi/\beta$, we have
\begin{equation}
Z_i \approx \ddfrac{1}{1 - \frac{\Im (\Sigma_i (i \omega_0))}{\omega_0}} = \ddfrac{1}{1 - \frac{\beta}{\pi} \Im (\Sigma_i (i \omega_0))}.
\end{equation}
To get the spectral function from the calculation of the Green's function on the Matsubara (imaginary) axis, we used the maximum entropy method for analytic continuation.

The in-house DMFT + QMC code used for this project has two main parameters to ensure convergence: the number of DMFT iterations $n_{\text{DMFT}}$ and the time the QMC solver runs for each DMFT iteration, $t_{\text{QMC}}$.
Other parameters worth mentioning are:
\begin{itemize}
\item The number of cores used for parallelisation, which is bound by the power of the computer used. We set this to 25 cores.
\item \sloppy{The thermalization time in the QMC solver, which has to be around one fourth of the total QMC time to ensure the system is well equilibrated before measurements are taken.}
\item The number of imaginary time slices in the QMC solver, which we set to 1024 slices to ensure a good resolution in imaginary time.
\end{itemize}
To choose appropriate values for $n_{\text{DMFT}}$ and $t_{\text{QMC}}$, we examined spectral function plots and monitored the change in quasiparticle weight as these parameters were increased.
Playing around with these parameters, we found that increasing both parameters improved convergence at roughly the same rate. Our convergence showed that even very large values
of $n_{\text{DMFT}}$ and $t_{\text{QMC}}$ improved the results (see figure \ref{fig:convergence}), so we set them to the maximum values our time constraints allowed: $t_{\text{QMC}} = 200\ \mathrm{s}$ and $n_{\text{DMFT}} = 200$.
You can also see the improvement in the spectral function on figure \ref{fig:spectral_improvement}.

\begin{figure}
    \centering
    \includegraphics[width=0.7\textwidth]{figures/convergence}
    \caption{Convergence of the quasiparticle weight as a function of DMFT iterations and QMC time. The number of DMFT iterations and the number of seconds for the QMC time are the same and are varied from 20 to 200.}
    \label{fig:convergence}
\end{figure}

\begin{figure}
    \centering
    \begin{subfigure}{0.49\textwidth}
        \centering
        \includegraphics[width=\textwidth]{figures/conv_low.pdf}
    \end{subfigure}
    \hfill
    \begin{subfigure}{0.49\textwidth}
        \centering
        \includegraphics[width=\textwidth]{figures/conv_high}
    \end{subfigure}
    \caption{Spectral function for $U=2.5$ at low (left) and high (right) convergence parameters.}
    \label{fig:spectral_improvement}
\end{figure}

The in-house DMFT + QMC code used for this project has two main parameters to ensure convergence: the number of DMFT iterations $n_{\text{DMFT}}$ and the time the QMC solver runs for each DMFT iteration, $t_{\text{QMC}}$.
Other parameters worth mentioning are:
\begin{itemize}
\item The number of cores used for parallelisation, which is bound by the power of the computer used. We set this to 25 cores.
\item \sloppy{The thermalization time in the QMC solver, which has to be around one fourth of the total QMC time to ensure the system is well equilibrated before measurements are taken.}
\item The number of imaginary time slices in the QMC solver, which we set to 1024 slices to ensure a good resolution in imaginary time.
\end{itemize}
To choose appropriate values for $n_{\text{DMFT}}$ and $t_{\text{QMC}}$, we examined spectral function plots and monitored the change in quasiparticle weight as these parameters were increased.
Playing around with these parameters, we found that increasing both parameters improved convergence at roughly the same rate. Our convergence showed that even very large values
of $n_{\text{DMFT}}$ and $t_{\text{QMC}}$ improved the results (see figure \ref{fig:convergence}), so we set them to the maximum values our time constraints allowed: $t_{\text{QMC}} = 200\ \mathrm{s}$ and $n_{\text{DMFT}} = 200$.
You can also see the improvement in the spectral function on figure \ref{fig:spectral_improvement}.

\begin{figure}[h!]
    \centering
    \includegraphics[width=0.7\textwidth]{figures/convergence}
    \caption{Convergence of the quasiparticle weight as a function of DMFT iterations and QMC time. The number of DMFT iterations and the number of seconds for the QMC time are the same and are varied from 20 to 200.}
    \label{fig:convergence}
\end{figure}

\begin{figure}[h!]
    \centering
    \begin{subfigure}{0.49\textwidth}
        \centering
        \includegraphics[width=\textwidth]{figures/conv_low.pdf}
    \end{subfigure}
    \hfill
    \begin{subfigure}{0.49\textwidth}
        \centering
        \includegraphics[width=\textwidth]{figures/conv_high}
    \end{subfigure}
    \caption{Spectral function for $U=2.5$ at low (left) and high (right) convergence parameters.}
    \label{fig:spectral_improvement}
\end{figure}

\section{Results}

The quasiparticle weight is plotted on Fig. \ref{TODO}.
The graph unambiguously shows an OSMT, with the narrow band becoming insulating at $U_{c1} \approx 2.1$ and the wide band at $U_{c2} \approx 3.1$. These results are to be compared with that of Koga's paper \cite{Koga2004} and Arita's paper \cite{Arita2005}, which show respectively $U_{c1} \approx 2.6, U_{c2} \approx 3.5$ and $U_{c1} \approx 2.6$

\begin{figure}[h!]
    \centering
    \includegraphics[width=\textwidth]{figures/z_full_J_all}
    \caption{Quasiparticle weight as a function of $U$ for $J=0.25U$. The critical parameter $U_c$ at which the Mott transition occurs is defined as the point where $Z$ reaches 0.}
    \label{fig:Z_full_J}
\end{figure}

Overall, our results are lower than those cited above. This might be agreement with theory, predicting that the critical $U$ at which transition occurs goes down as temperature is increased. \cite{Inaba2005} %Really ? Source ? Actually this might have nothing to do with temperature

Koga et al. argue that the isotropy in the full $J$ model is necessary to obtain an OSMT, explaining the absence of OSMT in Liebsch's results \cite{Koga2005}. In contrast, Knecht et al. \cite{Knecht2004} find lower values in the $J_z$ model, with $U_{c1} \approx 2.1$ and $U_{c2} \approx 2.6$, contradicting Liebsch's finding of the absence of OSMT \cite{Liebsch2004}. 

Actually, though, Liebsch published an answer to Knecht, showing that they do in fact obtain the same results and that the difference is mainly in interpretation of the results \cite{Liebsch2005}. Liebsch argues that although the spectral weights at $\omega = 0$ go to zero at different values of U, a noticeable kink is present when observing the quasiparticle weight of the wide band. This behaviour is something we also observe suprisingly well in Fig. \ref{Z_full_J}.

Liebsch argues this kink shows that there is a single transition, but that the narrow band does not transition to a full Mott insulator and rather to a bad metal. Seeing that there are few results that study the full $J$ model at finite temperature, and that bad metals are a thermal phase.

To further ensure our results are correct, we plot the full spectral function for a small set of values of $U$ and assess whether they display the characteristic behaviour of Mott transitions.

% Insert plot of spectral functions for 2 bands and at least two values of U

% Compare those values with those of Arita et al. They plot spectral functions for U = 2.0, 2.2, 2.4, 2.6

We further study the importance of the Hund coupling $J$ by running all the calculations again and setting $J = 0$ this time. The transition value is still not attained at $U = 5.5$, which is the maximum we could attain with our method before convergence issues.
The graph seems to be in agreement with Koga's results at $T = 0$ for the same model, where there is a single transition at $U_c = 7.3$, and $Z$ is at about $0.2$ for $U=5.5$.

\begin{figure}[h!]
    \centering
    \includegraphics[width=\textwidth]{figures/z_no_J_all}
    \caption{Quasiparticle weight as a function of $U$ for $J=0$.}
    \label{fig:Z_no_J}
\end{figure}

\section{Conclusion}

We considered a two-band Hubbard model with two different bandwidths \(W_1=2\)eV and \(W_2=4\)eV at room temperature \(T=1/40\)eV. The full \(J\) model was used, and the interaction strengths fixed at \(U'=U/2\) and \(J=U/4\). The self energy and the spectral function were calculated using DMFT with a QMC impurity solver. The behaviour of the spectral weight at the Fermi level shows an orbital selective Mott transition. The system undergoes two phase transitions, first at lower \(U\) for the narrow band, and at higher \(U\) for the wide one. Between the two critical values of \(U\), we obtain an intermediate phase, where the narrow band is insulating and the wide one is still conducting. These findings are in agreement with Knecht et al., who found it for the \(J_z\) model at finite temperature, with Koga et al., who found it for the full \(J\) model at 0 temperature and Arita and Held, who found it in both models at 0 temperature. In our work, we showed in addition that the transition remains orbital dependent at finite temperature for the full \(J\) model (table \ref{table_compare}). 
%Analyze what's missing in Liebsch's papers
%Compare relevance of results with other papers (is the finite T qualitatively different from 0 T, is it more realistic ; what does the J model add to the Jz model)

\begin{table}[h!]
\centering
\begin{tabular}{|p{3.5cm}|p{2cm}|p{1.4cm}|p{1.4cm}|p{1.4cm}|p{1.4cm}|}
\hline
\multicolumn{2}{|c|}{ } & \multicolumn{2}{|c|}{$J_z$ model} & \multicolumn{2}{|c|}{$J$ model}\\
 \hline
Author & T (eV)  & $U_{c1}$ (eV) & $U_{c2}$ (eV) & $U_{c1}$ (eV) & $U_{c2}$ (eV) \\
 \hline
 Liebsch, 2004 & 1/32 & 2.5 & 2.5 & & \\
 Koga et al., 2005 & 0 & 2.3 & 2.3 & 2.6 & 3.5 \\
 Knecht et al., 2005 & 1/40, 1/32 & 2.1 & 2.6 & & \\
 Arita, Held, 2005 & 0 & & & 2.6 & \\
 Our results & 1/40 & & & 2.1 & 3.1 \\
 \hline
\end{tabular}
\caption{Table comparing numerical values for Mott transitions. Liebsch's first paper is omitted as the parameters used were too different. Our results are}
\label{table_compare}
\end{table}

\section*{Acknowledgements}

\bibliographystyle{unsrt}
\bibliography{biblio}

\end{document}

\end{document}