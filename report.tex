\documentclass[titlepage, 11pt, a4paper]{article}
\usepackage[T1]{fontenc}
\usepackage{graphicx}
\usepackage{float}
\usepackage{tcolorbox}
\usepackage{amsmath}
\usepackage{amsfonts}
\usepackage{caption}
\usepackage{subcaption}

%Instructions : 5 pages, maximum 8 including figures

\newcommand{\cdag}[1]{\hat{c}_{#1}^{\dagger}}
\renewcommand{\c}[1]{\hat{c}_{#1}}
\newcommand{\n}[1]{\hat{n}_{#1}}

\begin{document}

\section{Introduction}

Dynamical Mean-Field theory (DMFT) has proven to be a fantastic tool to study the Hubbard model and Mott-Hubbard transitions. 
Still, the nature of Mott-Hubbard transitions is complex as we get away from the most simple models. Notably, in the years 2003 to 2005, a controversy was sparked regarding the existence of Orbital-Selective Mott Transitions (OSMTs). 
Indeed, in many transition metal oxides, multiple bands typically cross the Fermi surface, notably the $t_{2g}$ or $e_g$ orbitals, and different bandwidths for different bands might give rise to multiple transition for each bands. 
This is the case for the cuprate ${\mathrm{Ca}}_{2-x}{\mathrm{Sr}}_x{\mathrm{RuO}}_4$ which undergoes a Mott transition as $x$ is increased from 0 \cite{Nakatsuji2000}.

The model considered in this paper will be the case of two bands, one narrow ($W = 2 eV$)  and one wide band ($W = 4 eV$). The Hamiltonian used will be the Hubbard hamiltonian with interband coupling accounting for Hund's exchange :

\begin{align*}
H = &- \sum_{\langle i,j \rangle m \sigma} t_m \cdag{im\sigma} \c{jm\sigma} \\
&+ U \sum_{i m \sigma} \n{im\uparrow} \n{im\downarrow} + \sum_{i \sigma \sigma'} \left( U' - \delta_{\sigma \sigma'} J_z \right) \n{i1\sigma} \n{i2\sigma'} \\
&+ \frac{J_{\perp}}{2} \sum_{i m \sigma} \cdag{im\sigma} \left( \cdag{i \bar{m} \bar{\sigma}} \c{im\bar{\sigma}} + \cdag{im\bar{\sigma}} \c{i \bar{m} \bar{\sigma}} \right) \c{i \bar{m} \sigma}
\end{align*}

This model takes into account single-band Coulomb repulsion with the term in $U$, interband repulsion with $U' = U - 2J_z$ and Hund's exchange may be kept asymmetric ($J_z$ model) or simplified with $J_z = J_{perp} = J$ if the system is invariant under spin rotation ($J$ model). Sum indices denote sites ($i,j$), band ($m=1,2$) and spin ($\sigma = \uparrow, \downarrow$) with bars indicating the opposite state for the two states variables.

Initial results from Liebsch refuted the existence of OSMT for the $J_z$ model using a DMFT method with Quantum Monte-Carlo (QMC) solver at finite temperature \cite{Liebsch2003} \cite{Liebsch2004}. 
However, multiple papers later contradicted Liebsch's results and found an OSMT, first Koga et al. using DMFT with exact diagonalisation (ED) at zero temperature and the full $J$ model \cite{Koga2004}, then Knecht et al. using DMFT with an improved QMC solver attempting to get rid of the sign problem at low temperature \cite{Knecht2005}, using the same $J_z$ model asq Liebsch.
Finally Arita and Held also found an OSMT using projective QMC (PQMC) to tackle the $J$ model at zero temperature \cite{Arita2005}.

\section{Methods}

Our goal in this paper is to use DMFT with a QMC solver to clarify the controversy and ensure the two-band Mott transition predictions is well understood and well described with current methods. 
To that end, we choose to study the full $J$ model, which supposedly encompasses the full effects of correlation, and is also more interesting as it has only been studied at 0 temperature, contrary to the $Jz$ model. 
In the Hamiltonian, $t_m$ is set to 1 eV to set the energy scale. The parameters are chosen in accordance with previous papers on the matter, with the value of $U$ variable and the ratios $J = U/4$ and $U' = U - 2J = U/2$ fixed ; additionally the two bands are set with bandwidths $W_1 = 2 eV$ and $W_2 = 4 eV$ in an elliptical density of states given by $\rho_i (\epsilon) = \frac{4}{\pi W_i} \sqrt{1 - 4 \epsilon^2 / W_i^2}$, and taken at half filling.
Finally temperature is set to $\beta = \frac{1}{40} eV$, about room temperature.

To characterize the Mott transition, the parameter of choice is the quasiparticle weight, obtained from the self-energy :
\[ Z_i = \frac{1}{1 - \frac{\partial}{\partial \omega} \Re (\Sigma_i) \biggr\rvert_{\omega = 0}}\]

%TODO
%explain :
% observables used to characterize OSMT (quasiparticle weight)
% observables used to check convergence (spectral function)
% parameters of DMFT and convergence tests

\section{Results}

The quasiparticle weight is plotted on Fig. \ref{TODO}.
The graph unambiguously shows an OSMT, with the narrow band becoming insulating at $U_{c1} \approx 2.1$ and the wide band at $U_{c2} \approx 3.1$. These results are to be compared with that of Koga's paper and Arita's paper, which show respectively $U_{c1} \approx 2.6, U_{c2} \approx 3.5$ and $U_{c1} \approx 2.6$

Overall, our results are close to but lower than that cited above. This is in good agreement with theory, predicting that the critical $U$ at which transition occurs goes down as temperature is increased. %Really ? Source ? Actually this might have nothing to do with temperature

%TODO
%compare results for the transition values to the articles but especially Koga and Arita which have the same hamiltonian, although 0 T

\section{Conclusion}

%OSMT exists
%Analyze what's missing in Liebsch's papers
%Compare relevance of results with other papers (is the finite T qualitatively different from 0 T, is it more realistic ; what does the J model add to the Jz model)

\begin{table}[h!]
\centering
\begin{tabular}{|p{3.5cm}|p{2cm}|p{1.4cm}|p{1.4cm}|p{1.4cm}|p{1.4cm}|}
\hline
\multicolumn{2}{|c|}{ } & \multicolumn{2}{|c|}{$J_z$ model} & \multicolumn{2}{|c|}{$J$ model}\\
 \hline
Author & T (eV)  & $U_{c1}$ (eV) & $U_{c2}$ (eV) & $U_{c1}$ (eV) & $U_{c2}$ (eV) \\
 \hline
 Liebsch, 2004 & 1/32 & 2.5 & 2.5 & & \\
 Koga et al., 2003 & 0 & & & 2.6 & 3.5 \\
 Knecht et al., 2005 & 1/40, 1/32 & 2.1 & 2.6 & & \\
 Arita, Held, 2005 & 0 & 2.1 & & 2.6 & \\
 Our results & 1/40 & & & 2.1 & 3.1 \\
 \hline
\end{tabular}

\label{table_compare}
\caption{Table comparing numerical values for Mott transitions. Liebsch's first paper is omitted as the parameters used were too different. Our results are}
\end{table}

\section*{Acknowledgements}

\bibliographystyle{unsrt}
\bibliography{biblio}

\end{document}